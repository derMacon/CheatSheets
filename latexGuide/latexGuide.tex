\documentclass[11pt]{article}
\usepackage[utf8]{inputenc}
\usepackage[english, ngerman]{babel}
\usepackage[T1]{fontenc}
\usepackage{lmodern}


\title{Latex Zusammenfassung}
\author{Silas}
\date{03/14/15}

\begin{document}
\maketitle
Uebersicht ueber diverse Latex-Befehle und wie man diese benutzt.

\textbf{Alle} Befehle \emph{beginnen} mit einem Backslash

\section{Textuelle Darstellung}
\begin{itemize}
  \item \verb|\emph{...}|: hervorgehobener \emph{Text} (Besser als kursiv da Befehl verschachtelt aufgerufen werden kann)
  \item \verb|\textit{...}|: Kursiver \textit{Text}
  \item \verb|\textbf{...}|: Fett gedruckter \textbf{Text}
  \item \verb|\verb|: Quellcode font (quotet alle Latex-Befehle)
  \item \verb|\begin{verbatim*} <Quellcode> \end{verbatim*}|: Darstellung eines gesamten Codeblocks
\end{itemize}

\section{Formatierung}
\begin{itemize}
  \item Listen: 
  \begin{itemize}
  \item geordnet: Beginn mit begin{enumerate}, durchnummerierte Elemente. Elemente mit item ... bezeichnet
  \item ungeordnet: Beginn mit begin{itemize}, Elemente mit item ... angefuehrt
\end{itemize}

\item Ueberschriften
Koennen auch zur Generierung eines Inhaltsverzeichnisses genutzt werden. 
  \item Hauptueberschrift (toplevel): \emph{section}
  \item Unterueberschriften: \emph{subsection}
  \item Unterunterueberschriften: \emph{subsubsection} (tiefer geht es nicht)
  \\Mit dem Zusatz eines Sterns kann die Nummerierung der Ueberschriften vernachlaessigt werden (wird dann aber auch nicht im Inhaltsverzeichnis aufgefuehrt).
  Beispiel: 
  \begin{verbatim*} 
\section{Ueberschrift mit Nummerierung}
\section*{Ueberschrift ohne Nummerierung}
  \end{verbatim*}

  \item \verb|\quad <Text>|: Der Text wird per Tabulator eingerueckt
  \item \verb|\\|: Zeilenumbruch, ansonsten eine neue Leerzeile fuer Paragraphen einfuehren.
\end{itemize}

\subsection{Packages}
Befehl zum einbinden eines Paketes: \verb|\usepackage[<Optionale Parameter>]{<Verpflichtende Parameter}|.
\begin{itemize}
  \item \verb|\usepackage[utf8]{inputenc}|: Deutsche Umlaute werden verfuegbar gemacht emph{(Inputencoding)}
  \item \verb|\usepackage[Tl]{fontenc}|: Kodierung in der PDF-Datei, Umlaute koennen gesucht werden. 
  \item \verb|\usepackage[english, ngerman]{babel}|: Rechtschreibpruefung. Auch mit mehreren Sprach moeglich, letztgenannte immer die Hauptsprache
  \item \verb|\usepackage{lmodern}|: Schoener aussehende Fonts
\end{itemize}

\subsection{Klassen}
\begin{itemize}
  \item \verb|documentclass[a4paper]{scrartcl}|: Erneuerte Standarddokumentklasse mit unterschiedlichen Schriften.
\end{itemize}

\subsection{Metadaten}
Erst muessen die Metadaten deklaraiert werden. Dazu einfach die Befehle title, author und date aufrufen. 
\begin{itemize}
  \item \verb|\maketitle|: Muss in einem Dokument aufgerufen werden und erzeugt ein Deckblatt mit den gegebenen Metadaten.
  \item \verb|\thispagestyle{...}|: Der Parameter empty loescht saemtliche Seitenzahlen. 
\end{itemize}

\subsection{Sonstiges}
\begin{itemize} 
  \item Prozentzeichen zum Auskommentieren 
\end{itemize}


\end{document}
